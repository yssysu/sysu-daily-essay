% -------------------------------------------------------
% 样式设置
% -------------------------------------------------------
% 设置章节标题格式(使用宋体小四及以上)
\titleformat*{\section}{\songti\sanhao\bfseries\raggedright} % 一级标题 宋体三号加粗
\titleformat*{\subsection}{\songti\sihao\bfseries\raggedright} % 二级标题 宋体四号加粗
\titleformat*{\subsubsection}{\songti\xiaosi\bfseries\raggedright} % 三级标题 宋体小四加粗

% 定义居中章节标题命令(用于需要居中显示的特殊情况)
\newcommand{\centersection}{\titleformat*{\section}{\songti\sanhao\bfseries\centering}}

% 定义在目录与章节间添加横线的命令
\newcommand{\addhruleatend}{%
  \addtocontents{toc}{%
    \protect\vspace{1em}%
    \protect\noindent\protect\hrule%
    \protect\vspace{1em}%
  }%
}

% 美化目录条目样式
\renewcommand{\cftsecfont}{\songti\sihao\bfseries} % 一级目录为宋体四号加粗
\renewcommand{\cftsecpagefont}{\songti\sihao}
\renewcommand{\cftsubsecfont}{\songti\sihao} % 二级目录为宋体小四号
\renewcommand{\cftsubsecpagefont}{\songti\sihao}
\renewcommand{\cftsubsubsecfont}{\songti\xiaosi} % 三级目录也使用宋体小四号
\renewcommand{\cftsubsubsecpagefont}{\songti\xiaosi}

% 调整目录间距
\setlength{\cftaftertoctitleskip}{0.5em} % 目录标题与内容的间距
\setlength{\cftsecindent}{0em} % 一级目录缩进
\setlength{\cftsubsecindent}{1em} % 二级目录缩进
\setlength{\cftsubsubsecindent}{2em} % 三级目录缩进

% 设置目录标题居中并使用三号加粗
\renewcommand{\cfttoctitlefont}{\hfill\songti\sanhao\bfseries}
\renewcommand{\cftaftertoctitle}{\hfill}


% 为目录添加点线
\renewcommand{\cftdotsep}{1} % 设置点的间距
\renewcommand{\cftsecleader}{\cftdotfill{\cftdotsep}} % 一级目录添加点线
\renewcommand{\cftsubsecleader}{\cftdotfill{\cftdotsep}} % 二级目录添加点线
\renewcommand{\cftsubsubsecleader}{\cftdotfill{\cftdotsep}} % 三级目录添加点线


% -------------------------------------------------------
% 页面布局设置
% -------------------------------------------------------
% 页面尺寸和边距
\geometry{
    a4paper,
    left=2.5cm,   % 左边距
    right=2.5cm,  % 右边距
    top=2.5cm,    % 上边距
    bottom=2.5cm  % 下边距
}

% 行距设置
\setstretch{1.5} % 设置行距为 1.5 倍

% 页眉页脚设置,不需要的话也可以注释掉
\pagestyle{fancy}
\fancyhf{}
\fancyhead[L]{猎魔人学习笔记} % 左侧页眉
\fancyhead[R]{狼派猎魔人} % 右侧页眉
\fancyfoot[C]{\thepage} % 页脚居中显示页码

% 设置段落缩进
\setlength{\parindent}{2em} % 缩进两个汉字
\setlength{\parskip}{0em} 